\section{Perspektivering}
I dette projekt har vi forsøgt at løse fysiske og kodemæssige problemstillinger, med en agile fremgangsmåde. Omend dette ikke altid er lykkedes, har vi lært hvilke faldgrupper og brugbare processer, der kan bruges som udgangspunkt i fremtidige software opgaver og projekter.
\newline
De problemer vi er stødt på, som kan optimeres, er blandt andet vores brug af Goertzel algoritmen, som ikke bruger thresholds korrekt. Dette har givet problemer med støj, som videre har givet problemer med hastighed af besked afsendelse/modtagelse. Yderligere ville den næste iteration involvere refakturering af de mere komplekse dele af koden, deriblandt transport laget's modtagerside.
\\
Den næste iteration ville formodentlig indholde tråde, hvilke ville give nye problemstillinger og muligheder. Bl.a. ville vi være i stand til at lave dublex kommunikation, hvilket ville betyde at transportlaget skulle ændres til at include piggy-backing.
\\
Brugen af Scrum har haft en væsentlig indflydelse på den måde gruppemedlemmerne har kommunikeret med hinanden på, men samtidig har vi indset at det ikke er nok til at arbejde agilt og at et undervisende projekt af folk, der ikke er vant til at arbejde sammen som en gruppe, da agilt arbejde fungerer bedst når der arbejdes med noget, mindst én i gruppen har kompetancer til. Desuden fungerer agilt arbejde bedst i grupper hvor alle kender deres rolle.


%optimering i forhold i release, release er hurtigere end debug