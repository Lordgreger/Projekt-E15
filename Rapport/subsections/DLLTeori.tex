\subsection{Teori}
\underline{\textbf{CRC}}
\newline
CRC står for Cyklisk Redundant Check, og bruges til fejl-detektering. Denne metode bruges generelt både til LAN og WAN netværk.
\newline
\textbf{Modulu-2 Binary Division}
\newline
Modulo-2 binary division benytter XOR regnemetoden.
\newline
\hfill \break
\underline{\textbf{Stuffing}}
\newline
Bitstuffing tilføjer ubrugte bits for at gøre datastrukturer nemmere at manipulere.
\newline
Stuffing tilføjes bitstrengen efter resten er fundet, ved hjælp af modulus. Hvis resten er 0 tilføjes "100", hvis resten er 1 tilføjes "10" og hvis resten er 2 tilføjes "1"
giver mening i forhold til dataformat.