\subsection{Teori}
\underline{\textbf{CRC}}
\newline
CRC står for Cyklisk Redundant Check, og bruges til fejl-detektering. Denne metode bruges generelt både til LAN og WAN netværk.
\newline
\textbf{Modulo-2 Binary Division}
\newline
Modulo er resten af en division mellem to tal, og er oftest udtrykt som "\%".
\newline
I division fås udtrykket:
$$s = n \times d + r$$
\newline
Hvor s er dividenden, n er kvotienten, d er divisoren og r er resten.
\newline
I modulo defineres nye oprationer, som ofte ligner Boolske logiske operationer, heriblandt XOR, som bruges til addition, og AND, som bruges til mulitplikation.
Bitvist er er modulo-2 det samme som XOR, og her noteres modulo-2 additions operationen med en cirkel med et plus, f.eks.: 
$$1 \oplus 0 = 1$$

\newline
\hfill \break
\underline{\textbf{Stuffing}}
\newline
Bitstuffing tilføjer ubrugte bits for at gøre datastrukturer nemmere at manipulere.
\newline
Stuffing tilføjes bitstrengen efter resten er fundet, ved hjælp af modulus. Hvis resten er 0 tilføjes "100", hvis resten er 1 tilføjes "10" og hvis resten er 2 tilføjes "1"
giver mening i forhold til dataformat.